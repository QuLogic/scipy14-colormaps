\documentclass[11pt]{article}

\title{Perceptions of Matplotlib colormaps}
\author{Kristen M. Thyng}
\date{}


\begin{document}

\maketitle

The choice of colormap in a scientific figure significantly affects the way the presented information is perceived by the viewer. This talk will follow up on Damon McDougall's talk on how to choose a colormap for an  application by delving deeper into several important issues in regard to many of the available Matplotlib colormaps. It is known that people are best able to interpret changes in the luminance and saturation of colors in colormaps. Also, some research has shown that logarithmic changes in brightness are perceived as linear changes. Being able to print a color plot in black and white from a published paper is sometimes mandatory and often desirable, and is related to the grey scale in a colormap. Finally, it is important to remember various types of color blindness when choosing a divergent colormap. All of these concerns will be examined in the context of the available Matplotlib colormaps in order to better choose the best colormap for a given application.


Note: this abstract is intended to be for a talk in the Visualization mini symposium and to directly follow Damon McDougall's talk on colormaps.

\end{document}